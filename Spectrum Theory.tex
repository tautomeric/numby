\documentclass[12pt]{scrreprt}
 \usepackage{amsmath}
\title{Spectrum Theory}
\author{Thomas Burlingame-Smith}
 \date{Ongoing}
\begin{document}
\maketitle
\chapter{Spectrum Basics}
\section{What is a Spectrum?}
          Spectra are portions of the number line between a number point \textbf{a} and a number point \textbf{b} whereby only values where the difference between a given value in the sequence and the next value equal a quantum \textbf{q} are included.
Thus a spectrum can be represented as a sequence in a tuple of the form:
                                                                                            \begin{equation}  (a,a+q,a+2q,...,b)\end{equation}
when the quantum is constant. As seen above,such spectra follow a simple algebraic pattern.  
\section{The Quantum}
	The quantum may change according to a function of form \textbf{q=f(i)}, where\textbf{ i$ \geq$ 0} ,\textbf{ f(0) = 0} and correspond to \textbf{a}, and all values in the domain and range of \textbf{f(i)} are positive. Such spectra can be represented by the tuple of the form:
                                                                                                   \begin{equation}  (a+f(0),a+f(0)+f(1),a+f(0)+f(1)+f(2),...,b)\end{equation}
or alternatively:
                                                                                                    \begin{equation}  (a,a+f(1),a+f(1)+f(2),...,b)\end{equation}
\section{Quantum Truncating}
Quantum truncation adjusts the number line to a spectrum. 
The quantum floor for a number n is notated as: \textbf{$\lfloor$n$\rfloor^q$}\newline
Let set \textbf{s} be the set of numbers and \textbf{j} an element of \textbf{s} such that \textbf{q$|$j}:\begin{center} \textbf{$\lfloor$n$\rfloor^q $ = n } when \textbf{n$\in$ s}.\end{center}\begin{center} -Otherwise \textbf{$\lfloor$n$\rfloor^q $} is the smallest value  \textbf{j} such that \textbf{j $\geq$n}\end{center}
Meanwhile, the quantum ceiling for a number n is notated as: \textbf{$\lceil$n$\rceil^q$}\newline
\begin{center} \textbf{$\lceil$n$\rceil^q $ = n } when \textbf{n$\in$ s}.\end{center}\begin{center} Otherwise \textbf{$\lceil$n$\rceil^q $} is the largest value  \textbf{j} such that \textbf{j $\leq$n}\end{center}\begin{center} -In both these definitions it is assumed that \textbf{a $\leq$ n $\leq$ b}\end{center}
The number line may be shifted relative a spectrum. In which case, a shifted quantum ceiling and floor, denoted by \textbf{$\lceil$n$\rceil_a^q $} and  \textbf{$\lfloor$n$\rfloor_a^q $ } respectively, are used instead. In this case, every element in \textbf{s} is shifted to match the values on the spectrum of the denoted value \textbf{a} and quantum \textbf{q}.   
\subsection{Ceiling and Floor}
Note that: \begin{equation}
\textbf{$\lfloor$n$\rfloor$} = \textbf{$\lfloor$n$\rfloor^1$} 
\end{equation}
\begin{center}
Likewise:\end{center}
 \begin{equation}
\textbf{$\lceil$n$\rceil$} = \textbf{$\lceil$n$\rceil^1$} 
\end{equation}
\chapter{Functions Over a Spectrum}
\section{How Functions Run Over a Spectrum}
A function \textbf{f(x)} ran over a given spectrum containing the values in set \textbf{j} between \textbf{a} and \textbf{b} and of quantium \textbf{q} is represented as:
\begin{equation}[f(x)]_{a,b}^q
\end{equation}
Here, \textbf{f(x) =$ [f(x)]_{a,b}^q$} when \textbf{$f(x) \in j$} and \textbf{$[f(x)]_{a,b}^q$} is undefined otherwise.
\par However, if we want to run a function over a spectrum such that the full domain of \textbf{f(x)} appears in the spectrum, we can then express \textbf{f(x)} as  \textbf{$\lfloor$f(x)$\rfloor_a^q $} or \textbf{$\lceil$f(x)$\rceil_a^q $}.
\chapter{Overloaded Spectrum}
\section{Introduction}
A spectrum is overloaded when one describes numbers beyond the scope of the spectrum, such that when a number \textbf{k} is less than  \textbf{a} or greater than \textbf{b}. Any number described must still be a multiple of the quantum. 
\section{Notation and Descriptions}
The overloading of a spectrum where \textbf{k $>$ b}  is represent by $\aleph_n$ where \textbf{n} is the number of steps in the sequence above \textbf{b} the value  $\aleph_n$ would be if the spectrum extended to that value  $\aleph_n$. Meanwhile, overloading where  \textbf{k $<$ a} is represented by  $\aleph_n^-$ where \textbf{n} is the number of steps in the sequence below \textbf{a} if the spectrum had extended to $\aleph_n^-$.
\section{Overloaded Logic}
Overloaded logic is logic whereby something can exist as partially true and partially false. Overloaded logic can be represented by a spectrum of \textbf{a=0,b=1, q=1} with overloading occurring above \textbf{b} but not below \textbf{a}. \par0, as in non-overloaded logic, denotes something as false while greater values represent more trueness. 
\chapter{Overlayed Spectra}
\section{How Spectra Are Overlayed}
A spectrum \textbf{s$_n$} is overlayed on a spectrum \textbf{s$_{n-1}$} when  \textbf{s$_n$} is defined only when an event \textbf{e} occurs within a certain area on the spectrum \textbf{s$_{n-1}$}.A group of overlayed spectrum can be represented by a tuple of tuples as follows: 
   \begin{equation} ((s_1,d),(s_2,d),...(s_n))\end{equation},where \textbf{d} is the set of values on the spectrum which note when the next spectrum in the sequence can be defined.
\chapter{Quantum Geometry}
\section{Introduction}
A quantum geometry is a geometry where the possible values of all or a few measurements lay on a spectrum. Of course, this definition is wide open and allows for various possible geometries within those that already exist.
\section{Unfilled Quantum Geometry}
An unfilled quantum geometry requires one dimensional measurements to lay on a spectrum while allowing area to be divided via an uncountable infinite amount of times. 
\subsection{A \textit{Quantum} Thought Experiment and Unfilled Quantum Geometry}
In \textit{quantum} physics, particles are often quantified as probability spaces as classical physics does not accurately describe their motion. The minimum scale at which classical mechanics can still describe is the plank length which for our purposes be denote $\hbar$. 
\par If we want to consider possible distances over which an object may move within the expected bound of classical mechanics. Then, we can note a spectrum \textbf{s} of \textbf{q=$\hbar$} and bounds \text{a = $\hbar$} and \textbf{b = U}, where \textbf{U} is the quantum floor of the maximum length of the universe in multiples of $\hbar$. Any distanced traveled by a classical object must have then traveled a distance represented by this spectrum. 
\par From here, we can, then, describe allowed motion in any possible direction. We define these distances as possible line segment from which we then form 2D shapes. Since this geometry is unfilled, the shapes then can take on any area which may appear as long every line segment forming the shape is of a length that is a multiple of $\hbar$.
\section{Filled Quantum Geometry}
In a filled quantum geometry, aside from one-dimensional measurements, measurements of area, also, lay on a spectrum. The minimum length that a one-dimensional measurement can be in quantum space is always equivalent to the quantum. However, this is not always the case with area in a filled quantum geometry.  The number of steps in the sequence of the spectrum which must be transverse by the length of a line segment which forms the shape before the minimum area occurs is call the filling factor, \textbf{f}.
\subsection{Continuation of the \textit{Quantum} Thought Experiment}
\par Previously, we seek to described the limits of the intervals of motion in the macroscopic world. Now, we will describe the smallest possible sized objects of a Euclidean shape can exist in the microscopic world (we will ignore volume for now). Area will have to lay on a spectrum where units squared must be on the same spectrum as the units of length. The area of a rectangle is its length multiple by it width, and because any multiple inherits the factors of its factors, we can expect that any square formed from possible line segments in our filled quantum Euclidean geometry would be able to exist within it; therefore, \textbf{f=1}. 
\par Now let's look at the triangle. The area of a triangle is \textbf{.5Bh}. Now let's try to form a right triangle purely out of line segments with each segment of length $\hbar$. Such a triangle would have an area of $.5\hbar$ which is not allowed in our geometry. In order for the area of the right triangle to reach $\hbar$, at least one length must be 2$\hbar$; therefore \textbf{f=2}.
\section{Multi-Dimensional Fillness}
Any number of geometries may be formed with different axioms over which quantum may play a role. These geometries can be classified via fillness. For example, a four-dimensional geometry where area and hypervolume but line length and volume do not would have 2,4-fillness.

\end{document}